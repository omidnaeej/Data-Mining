
\documentclass[12pt]{article}

\usepackage{fancyhdr}
\pagestyle{fancy}
\rhead{Omid Naeej Nejad}
\lhead{Data Mining - HW1}

\usepackage{blindtext}
\title{Data Mining \\ HW 1}
\date{\today}
\author{Omid Naeej Nejad \\ 610301189}


\begin{document}
	\maketitle
	\section{Exercise 1}
		\begin{enumerate}
			\item Description
			
			The description involves summarizing and visualizing data to understand its main characteristics. e.g. plotting clustered bar chart to check interaction between side effects or effectiveness of a drug against patient's demographics.
			
			\item Estimation
			
			Estimating the length of hospital stay based on patient demographics and medical history can help in resource planning and management.
			
			\item Prediction
			
			Predicting patient outcomes, such as recovery times required after a specific surgery or improvement in health after a specific treatment administered, based on patient's demographics and medical history.
			
			\item Classification
			
			Categorize patients into different risk groups (low, medium, high) based on patient's demographics, medical history and diagnostic results. 
			
			\item Clustering
			
			Clustering patients with the most similarity in diagnostic results, patient demographics and medical history to identify people are at risk of diseases, optimizing administered treatments and hospital resources.
			
			\item Association
			
			Association identifies relationships between variables. For example, finding associations
			between certain treatments and patient outcomes can help in understanding the effectiveness of
			different treatment protocols and optimizing them for better results.

		\end{enumerate}
	
	
	\section{Exercise 2}
			\begin{enumerate}
				\item Description: 
				This task involves summarizing and visualizing the data to identify key trends and patterns in voter behavior. Description helps in understanding the overall voting trends, demographic influences, and other significant factors.
				
				\item Estimation \& Prediction: Estimation is used to predict a continuous value, such as average monthly revenue. This task involves analyzing historical sales data to estimate future revenue.
				
				\item If the data points are labeled, Classification, If not, Clustering.
				
				\subitem Classification assigns data into predefined categories. In this case, customers are categorized into different credit risk groups based on their financial history, which helps in risk management and decision-making.
				\subitem Clustering similar customers together without predefined categories. (Number of Clusters = 3)
				
				\item Clustering: Clustering groups similar data points together without predefined categories. This task helps in identifying patterns and subgroups of patients with similar symptoms, which can be crucial for diagnosing new diseases and treatments.
				
				\item Association: Association identifies relationships between variables. This task involves finding common combinations of products that are frequently bought together, which can help in optimizing product placement and marketing strategies.

				\item Prediction: Prediction involves forecasting future outcomes based on historical data. This task helps in predicting the likelihood of a car model being recalled, which can be crucial for quality control and risk management.
				

			\end{enumerate}

	
\end{document}
