
\documentclass[12pt]{article}
\usepackage[utf8]{inputenc}
\usepackage{amsmath}
\usepackage{fancyhdr}
\pagestyle{fancy}
\rhead{Omid Naeej Nejad}
\lhead{Data Mining \\ Chapter 4\&5 Exercises}

\usepackage{blindtext}
\title{Data Mining \\ Chapter 4\&5 Exercises}
\date{\today}
\author{Omid Naeej Nejad \\ 610301189}


\begin{document}
	\maketitle
	\section{Exercise 1}	
		\paragraph{Constructing the 95\% Confidence Interval}
			\begin{itemize}
				\item $\alpha$ = 0.05 $\rightarrow$ $t_{\alpha/2}$ = 1.96.

				\item Calculate the margin of error (ME):
				\[
				ME = t_{\alpha/2} \times \frac{\sigma}{\sqrt{n}} = 1.96 \times \frac{3}{\sqrt{30}} \approx 1.96 \times 0.548 \approx 1.074
				\]
				
				\item Construct the confidence interval:\\
				\[
				(\mu - ME, \mu + ME) = (15 - 1.074, 15 + 1.074) = \textbf{(13.926, 16.074)}
				\]
				\end{itemize}	
	\section{Exercise 2}
	\subsection{}
		Yes, we can conclude that the new drug significantly reduces blood pressure. A p-value of 0.03 indicates that there is a 3\% chance of observing  a more extreme effect if the null hypothesis (that the drug has no effect) were true. Since this p-value is typically below the conventional significance level of 0.05, we reject the null hypothesis and conclude that the drug is effective.
		
		\[
		H_0: \mu=0 
		\]
		\[
		H_1: \mu < 0
		\]
		where $\mu$ is the true mean change in blood pressure after taking the drug (in this case, a reduction). If $\mu=0$, it means there is no change, indicating the drug is not effective.
		
	\subsection{}
		The margin of error in this experiment represents the range within which the true average reduction in blood pressure is likely to occur. It provides a measure of the \underline{uncertainty or variability} associated with the estimated average reduction.\\
		In this case, with a margin of error of ±2 mmHg, we can be reasonably confident (95\%) that the true average reduction lies between 3 mmHg (5 mmHg - 2 mmHg) and 7 mmHg (5 mmHg + 2 mmHg).
	\subsection{}
		\textbf{Margin of error:} Increasing the sample size reduces the margin of error $\Rightarrow$ the confidence interval becomes narrower, and we can be more certain about the true value of the average reduction in blood pressure.\\
		\textbf{P-value:} Increasing the sample size generally increases the statistical power of the test $\rightarrow$ the p-value is likely to decrease		
	\section{Exercise 3}
		\paragraph{95\% Confidence Interval of the proportion}	
			\begin{itemize}
				\item $\alpha$ = 0.05 $\rightarrow$ $Z_{\alpha/2}$ = 1.96.
				
				\item Calculate the margin of error (ME):
				\[
				p = \frac{180}{1200} = 0.15
				\]
				\[
				ME = Z_{\alpha/2} \times \sqrt{\frac{p.(1-p)}{n}} = 1.96 \times \sqrt{\frac{0.15 \times 0.85}{1200}} \approx 1.96 \times 0.0103 \approx 0.0202
				\]
				
				\item Construct the confidence interval:\\
				\[
				(p - ME, p + ME) = (0.15 - 0.0202, 15 + 0.0202) = \textbf{(0.1298, 0.1702)}
				\]
			\end{itemize}	

	\section{Exercise 4}
	\paragraph{Hypothesis Test For Proportion}
	\[
	H_0: p \geq 0.2 \quad vs \quad H_1: p < 0.2
	\]

	\[
	n = 150, \quad x = 30, \quad \hat{p} = \frac{x}{n} = \frac{30}{150} = 0.2
	\]
	

	\subparagraph{Calculate the Test Statistic}

	The test statistic (Z-score) is given by:
	\[
	Z = \frac{\hat{p} - \pi_0}{\sqrt{\frac{\pi_0 (1 - \pi_0)}{n}}}
	\quad \pi_0 = 0.2 
	\]
	\[
	Z = \frac{0.2 - 0.2}{\sqrt{\frac{0.2 \times (1 - 0.2)}{150}}} = 0
	\]
	
	\subparagraph{Compare the p-value to the significance level ($\alpha$ = 0.05)}
	\[
	p-value=p(Z \leq Z_{data})=\Phi(0)=0.5
	\]
	
	%If the p-value is less than $\alpha$, reject the null hypothesis.
	%If the p-value is greater than or equal to $\alpha$, fail to reject the null hypothesis.   
	
	Since the p-value (0.5) is greater than $\alpha$ (0.05), so we fail to reject the null hypothesis $\rightarrow$  the population proportion of deactivated users is not less than 20\%.
	
	\section{Exercise 5}

		%\paragraph{Two-Sample t-Test for Mean Differences}
		\begin{align*}
			H_0 &: \mu_1 = \mu_2 \quad \text{(No difference in means)} \\
			H_1 &: \mu_1 \neq \mu_2 \quad \text{(Difference in means)}
		\end{align*}

		\[
		t = \frac{(\bar{x}_1 - \bar{x}_2)}{\sqrt{\frac{s_1^2}{n_1} + \frac{s_2^2}{n_2}}} = \frac{(35.2 - 34.8)}{\sqrt{\frac{6.1^2}{1500} + \frac{5.9^2}{700}}} \approx 1.465
		\]		
		
		\[
		t_{\alpha/2}=1.96 \\
		\]
		
		Since $|t| \not> t_{\alpha/2} \rightarrow$  fail to reject the null hypothesis, so the partition is valid.


	\section{Exercise 6}
	Null Hypothesis ($H_{0}$): The distribution of education levels is independent of the group (experimental vs. control).\\
	Alternative Hypothesis ($H_{1}$): The distribution of education levels is not independent of the group.
	

	\subparagraph{The Chi-Square test statistic is calculated as:}
	
	\[
	\chi^2_{data} = \sum \sum \frac{(O - E)^2}{E}
	\]
	
	 $O$ : observed frequency and $E$ : expected frequency in a cell


	\[
	E = \frac{\text{row total} \times \text{column total}}{\text{grand total}}
	\]
	

	\begin{align*}
		E(\text{Experimental, Below High School}) &= \frac{1200 \times 650}{1550} \approx 503.23 \\
		E(\text{Experimental, High School}) &= \frac{1200 \times 520}{1550} \approx 403.23 \\
		E(\text{Experimental, Bachelor's and Above}) &= \frac{1200 \times 380}{1550} \approx 293.55 \\
		E(\text{Control, Below High School}) &= \frac{350 \times 650}{1550} \approx 146.77 \\
		E(\text{Control, High School}) &= \frac{350 \times 520}{1550} \approx 116.77 \\
		E(\text{Control, Bachelor's and Above}) &= \frac{350 \times 380}{1550} \approx 86.45
	\end{align*}

	\begin{align*}
		\chi^2_{data} &= \frac{(500 - 503.23)^2}{503.23} + \frac{(400 - 403.23)^2}{403.23} + \cdots + \frac{(80 - 86.45)^2}{86.45} \\
		&\approx 1.97
	\end{align*}

	\[
	p-value = P(\chi^2 > \chi^2_{data}) = P(\chi^2 > 1.97) \approx 0.373
	\]
	Since the calculated p-value is greater than the $\alpha$, we fail to reject the null hypothesis, so the partition is valid.
		
	\section{Exercise 7}	
	Null Hypothesis ($H_0$): The population means for the three methods are equal. \\
	Alternative Hypothesis ($H_1$): At least one of the population means is different.

	\[
	\text{Mean of Website} = \frac{25 + 30 + 28 + 32}{4} = 28.75
	\]
	\[
	\text{Mean of Mobile App} = \frac{35 + 40 + 38 + 36}{4} = 37.25
	\]
	\[
	\text{Mean of In-Person} = \frac{40 + 45 + 50 + 48}{4} = 45.75
	\]
	
	\[
	\text{Total Mean} = \frac{25 + 30 + 28 + 32 + 35 + 40 + 38 + 36 + 40 + 45 + 50 + 48}{12} = 37.25
	\]

	\textbf{Sum of Squares Between Groups (SSB)}\\	
	The sum of squares between groups measures the variation between the means of each group:
	
	\[
	SSB = n \left( (\overline{X}_{\text{website}} - \overline{X}_{\text{total}})^2 + (\overline{X}_{\text{mobile}} - \overline{X}_{\text{total}})^2 + (\overline{X}_{\text{in-person}} - \overline{X}_{\text{total}})^2 \right)
	\]
	
	\[
	SSB = 4((28.75-37.25)^2 + (37.25-37.25)^2 + (45.75-37.25)^2) = 578
	\]
	
	\textbf{Sum of Squares Within Groups (SSW)}\\
	The sum of squares within groups measures the variation within each group:
	
	\begin{align*}
	SSW &= \sum_{i=1}^{k} \sum_{j=1}^{n} (X_{ij} - \overline{X}_{\text{group}})^2 \\
	&= 98.25
	\end{align*}
	
	\textbf{Degrees of Freedom \& Mean Squares}\\
	The degrees of freedom between groups is:
	\[
	df_{\text{between}} = k - 1 = 3 - 1 = 2
	\]
	The degrees of freedom within groups is:
	\[
	df_{\text{within}} = N - k = 12 - 3 = 9
	\]
	
	\[
	MSB = \frac{SSB}{df_{\text{between}}} = \frac{578}{2}=289
	\]
	
	
	\[
	MSW = \frac{SSW}{df_{\text{within}}}= \frac{98.25}{9} \approx 10.917
	\]
	
	\textbf{F-statistic}\\
	
	\[
	F = \frac{MSB}{MSW} = 26.47
	\]
	
	\[
	p-value = P(F > F_{data}) = P(F >26.47) \approx 0.00017
	\]
	
	Since the calculated p-value is less than the $\alpha$, we reject the null hypothesis, so  the population mean time spent for receiving services differs among the three methods.
	\section{Exercise 8}	
	
	Estimated score=20+3×(Number of study hours)
	
	\subsection{}	
	According to the regression equation, for each additional hour of study, the score increases by 3 points.\\		
	So, the student who studied 5 hours more would have an estimated score that is 15 points higher than the other student.
	
	\subsection{}	
	\[
	\text{Estimated score}=20+3 \times 10=50
	\]
	
	\subsection{}	
	The result might not be reliable or accurate for students outside range(5, 15) like this one:
	\[
	\text{Estimated score}=20+3\times20=80
	\]

	\subsection{}		
	This means that for each additional hour a student spends studying, the estimated score increases by 3 points. 
	
	\subsection{}		
	The 20 in the equation represents the y-intercept of the regression line or equation predicts that if a student does not study at all, the score would be 20. 
	
\end{document}
